\documentclass[]{article}
\usepackage{amssymb,amsmath}
\usepackage{ifxetex,ifluatex}
\usepackage{fixltx2e} % provides \textsubscript
\usepackage{float} % provides the H option for float placement
\usepackage{graphicx}
\ifxetex
  \usepackage{fontspec,xltxtra,xunicode}
  \defaultfontfeatures{Mapping=tex-text,Scale=MatchLowercase}
  \newcommand{\euro}{€}
\else
  \ifluatex
    \usepackage{fontspec}
    \defaultfontfeatures{Mapping=tex-text,Scale=MatchLowercase}
    \newcommand{\euro}{€}
  \else
    \usepackage[utf8]{inputenc}
  \fi
\fi
\usepackage{url}
\ifxetex
  \usepackage[setpagesize=false, % page size defined by xetex
              unicode=false, % unicode breaks when used with xetex
              xetex,
              bookmarks=true,
              pdfauthor={},
              pdftitle={},
              colorlinks=true,
              urlcolor=blue,
              linkcolor=blue]{hyperref}
\else
  \usepackage[unicode=true,
              bookmarks=true,
              pdfauthor={},
              pdftitle={},
              colorlinks=true,
              urlcolor=blue,
              linkcolor=blue]{hyperref}
\fi
\hypersetup{breaklinks=true, pdfborder={0 0 0}}
\setlength{\parindent}{0pt}
\setlength{\parskip}{6pt plus 2pt minus 1pt}
\setlength{\emergencystretch}{3em}  % prevent overfull lines
\setcounter{secnumdepth}{0}
\usepackage[vmargin=1in,hmargin=1in]{geometry} 


\begin{document}

\section{Assessing the suitability of forest inventories for basis of
spatial conservation prioritization}

Joona Lehtomäki\textsuperscript{1,2*} Sakari Tuominen\textsuperscript{3}
and Antti Leinonen\textsuperscript{4}

1 Department of Biosciences, P.O. Box 65 (Viikinkaari 1), FI-00014
University of Helsinki, Finland\\2 Finnish Environment Institute,
Natural Environment Centre, P.O. Box 140 (Mechelininkatu 34a), FI- 00251
Helsinki, Finland\\3 Finnish Forest Research Institute
(Metsäntutkimuslaitos), Vantaa, Finland\\4 The Finnish Forest Centre
(Suomen Metsäkeskus), Kajaani, Finland\\* Corresponding author

\textbf{Contact:}\\joona.lehtomaki@helsinki.fi, tel.
+358-9-191-57714\\sakari.tuominen@metla.fi\\antti.leinonen@metsakeskus.fi

\textbf{Journal:}\\Scandinavian Journal of Forest Research

\textbf{Type of paper:}\\Original research paper

\textbf{Running title:}\\Validation of spatial conservation
prioritization

\textbf{Manuscript statistics:}\\* word count (total) = xxx\\* figures =
xxx of which xxx in color\\* tables = xxx\\* references = xxx

\subsection{Abstract:}

Keywords: open data; adaptive management; conservation planning; forest
conservation management; spatial conservation prioritization; Zonation
software

\subsection{1. Introduction}

\subsection{2. Material and methods}

\subsection{3. Results}

\subsection{4. Discussion}

Open data has a major role in the play in conservation planning and
decision-making, because it enables equal access to best available data,
it makes the supporting scientific analysis more transparent, and it
enhances the repeatability of the whole process (\textbf{REF}). The last
point is especially important for applied research supporting
decision-making, because underlying may change objectives, data updates,
and new information accumulates sometimes rapidly. Adaptivity and
repeatability are crucial in translating regional planning into local
action (Pressey et al. 2013) which in the context of the current work
implies that the conservation prioritization methods would become an
integral part of operative forest management planning.

Here we have shown that Zonation analyses based on open forest inventory
data can produce informative results, but only on particular scales and
when particular planning objectives are met. (\textbf{Add} assumptions
about the accuracy of the data). The validation procedure we did relies
on few key assumptions. First, if the index layers constructed from
different input data sets truly reflect forest characteristics desirable
for conservation purposes, then the sites in the validation data sets
should receive high priority in the results. Second, we assume that the
validation data sets actually describe location of high conservation
value. Protected areas have traditionally been established on less
productive soils (Scott et al. 2001; Elbakidze et al. 2013) and
therefore it is likely that they usually do not represent the full
spectrum of species or habitats in any given region. However, over a
period of time being set-aside from the prevailing forest management
regimes will produce resources such as dead-wood that many forest are
depend on (Siitonen et al. 2000). Woodland key-habitats on the other
hand are scattered more evenly over the landscape and according to
recent meta-analysis (Timonen et al. 2011) they contain higher amounts
of critical resources such as dead-wood and consequently larger number
of species as well (\textbf{check} the details for Finland). On average
the size of a WKH site is very small (0.67 ha in Finland (Timonen et al.
2010)) and thus they're capability to support populations in the long
run in uncertain. METSO-programme has strict protocol for assessing the
suitability of each site (\textbf{REF}?) and the monitoring studies done
so far (\textbf{REF}?) have concluded that sites admitted into the
programme indeed do have high conservation values.

Protected areas (PAs) did come out with relatively high priorities in
variants based on MSNFI or MSNFI with classes (fig. 5A-5D). This is
probably because PAs in South-Savo region - as in rest of the country -
tend to have been set aside for a longer period of time resulting in
quite mature forest structure, which is reflected in high values in the
index layers (see XX) used as input. Sites acquired into METSO-programme
had also relatively high median priorities potentially indicating that
they contain similar features to the larger PAs. Woodland key-habitats
have the smallest average size per site out of the three validation data
sets and they constitute perhaps also the most heterogeneous
(\textbf{REF}) group. The fact that the classified version of MSNFI had
slightly higher median priorities (fig. 5i-5j) is most probably
explained by that many of the rarer soil fertility classes (such as
herb-rich and xeric soils) may also more often be designated as woodland
key-habitats. Variants based on the more detailed data (V5 and V6)
clearly outperform all variants based solely on the MSNFI data when
compared against the validation data sets. Median priority is clearly
higher for the variants based on the detailed data and furthermore, the
priority rank distribution shows conspicuous peaks for the very highest
priorities. Because of the more detailed and accurate data - also for
the soil fertility classification - analysis are able to distinguish
small-scale woodland key-habitats quite well.

The small effect of connectivity (V2, V4 and V6) has on the priority
rank distributions of the validation data sets may feel slightly
surprising. However, it is good to bear in mind that even combined the
validation data sets cover only a small fraction of the total landscape
(2.5\%, see XX) and therefore the absolute amount of cells affected by
connectivity transformation is low. Usually spatial conservation
prioritization is concerned about the absolutely best part of the
landscape, the defining what is the ``best part'' is a subjective
decision. For example, METSO-programme has a defined objective for
additional conservation in South-Savo, which correspond to ca. 5000
hectares or less than 0.5\% of the total forest area (\textbf{CHECK}).
On the other hand, if different conservation instruments are to employed
over a significantly larger areas (Hanski 2011; Moen et al. 2014), we
must in fact be looking at top fractions significantly higher than few
percent. Over larger areas, the role of connectivity also becomes more
apparent (fig. 2) as regions with higher density of high quality sites
are emphasized. Emphasizing connectivity will almost certainly happen at
the expense of local habitat quantity and quality (Hodgson et al. 2009).
Increasing the priority of medium-quality forests that are
well-connected will lower the value of other similar quality sites and
possibly even poorly connected high-quality sites (fig. 2). Trade-offs
introduced by taking into account connectivity will naturally depend on
the implementation of a particular method (in or case Zonation), but the
issues related to it are conceptually quite well understood (Hodgson et
al. 2009; Arponen et al. 2012).

Highest and lowest fractions of landscape seem to be consistently more
similar in terms of overlap measured by the Jaccard coefficient (fig.
3). It is not perhaps surprising given that the best parts of the
landscape probably are best by a large margin and the worst parts
probably do not have much forest at all. In none of the variant
comparisons do the best and worst parts of landscape overlap much. From
input data sets' perspective this can be considered a good thing,
because such overlaps would imply serious risks of selecting poor sites
if using the coarser data. Comparisons between the MSNFI and MSNFI with
classes input data sets (V1 vs.~V3 and V2 vs.~V4) reveals interesting
patterns caused by classification. (\textbf{EXPAND})

Trade-offs introduced by using less accurate data are more
case-specific, but it can be estimated. Conservation scientists,
foresters and other practitioners are often faced with tight deadlines
and limited budgets, and thus have to decide whether it is worth the
time and money to try to secure access to more detailed data if coarser
but easily available data exists. We found that using coarser MSNFI-data
can lead to a serious drop in the representation of especially the less
abundant biodiversity features such as the herb-rich and xeric forest
types (fig. 4). For example, if we are interested in the best 10\% of
the landscape prioritization based on MSNFI with classes captures, on
average, only half of the representation levels of the biodiversity
features derived from the detailed data. For biodiversity feature on
herb-rich soils, analysis based of MSNFI with classes captures less than
10\% (\textbf{check} the exact figures) of representation levels. These
differences are probably mostly due to less accurate soil fertility
classification in the MSNFI data (\ldots{}). For the rarest, and hence
most valuable, soil fertility classes (herb-rich and xeric) MSNFI with
classes performs slightly better than MSNFI without classes. Hence, an
ecologically justified classification of the data can improve the
results even if the quantitative information (i.e.~the index values) is
the same.

(A paragraph on the effects of segmentation?)

Given these results, it can be concluded that when using methodology we
introduce in this paper here and the variations of it used before
(Lehtomäki et al. 2009; Sirkiä et al. 2012) openly available MSNFI-data
is best suited for situations where objective is to target regions with
larger extents of mature forest. Therefore if the spatial prioritization
includes objectives for detecting small scale biodiversity feature
occurrences such as the WKHs, a more detailed input data set is clearly
needed. Given the amount of available forest inventory data in countries
like Finland and Sweden, it is very important that these data sources
can be used also for conservation planning purposes for several reasons.
First, since high-resolution, large-scale, and systematic observational
biodiversity data is scarce (\textbf{REF}) suitable proxies for species
and habitat occurrence are needed. Second, over 95\% of the forest
landscapes in Finland and Sweden is under silvicultural management
(\textbf{CHECK} METLA 2013; Skogstyrelsen XXXX) which uses these data
for operative planning. If conservation planning is to be integrated
with other types of land use and natural resource planning, it would be

From conservation planning perspective it is crucial that forest
inventory data sets collected contain a breadth of variables important
for biodiversity also in the future.

Since the analysis based on MSNFI data do not perform as well as more
detailed data\ldots{} Should the more detailed data be opened? Or at
least part of it.

\section{References}

Arponen, A., J. Lehtomäki, J. Leppänen, E. Tomppo, and A. Moilanen.
2012. Effects of connectivity and spatial resolution of analyses on
conservation prioritization across large extents. Conservation Biology
\textbf{26}:294--304. Retrieved from
\url{http://doi.wiley.com/10.1111/j.1523-1739.2011.01814.x}.

Elbakidze, M., P. Angelstam, N. Sobolev, E. Degerman, K. Andersson, R.
Axelsson, O. Höjer, and S. Wennberg. 2013. Protected area as an
indicator of ecological sustainability? A century of development in
Europe's boreal forest. Ambio \textbf{42}:201--14. Retrieved from
\url{http://dx.doi.org/10.1007/s13280-012-0375-1}.

Hanski, I. 2011. Habitat loss, the dynamics of biodiversity, and a
perspective on conservation. Ambio \textbf{40}:248--255. Retrieved from
\url{http://www.springerlink.com/index/10.1007/s13280-011-0147-3}.

Hodgson, J. A., C. D. Thomas, B. A. Wintle, and A. Moilanen. 2009.
Climate change, connectivity and conservation decision making: back to
basics. Journal of Applied Ecology \textbf{46}:964--969. Retrieved from
\url{http://blackwell-synergy.com/doi/abs/10.1111/j.1365-2664.2009.01695.x}.

Lehtomäki, J., E. Tomppo, P. Kuokkanen, I. Hanski, and A. Moilanen.
2009. Applying spatial conservation prioritization software and
high-resolution GIS data to a national-scale study in forest
conservation. Forest Ecology and Management \textbf{258}:2439--2449.
Retrieved from
\url{http://linkinghub.elsevier.com/retrieve/pii/S0378112709005969}.

Moen, J., L. Rist, K. Bishop, F. S. Chapin III, D. Ellison, H.
Petersson, K. J. Puettmann, J. Rayner, I. G. Warkentin, and C. J. A.
Bradshaw. 2014. Eye on the Taiga: Removing global policy impediments to
safeguard the boreal forest. Conservation Letters.

Pressey, R. L., M. Mills, R. Weeks, and J. C. Day. 2013. The plan of the
day: Managing the dynamic transition from regional conservation designs
to local conservation actions. Biological Conservation
\textbf{166}:155--169. Retrieved from
\url{http://linkinghub.elsevier.com/retrieve/pii/S0006320713002073}.

Scott, J. M., F. W. Davis, R. G. McGhie, R. G. Wright, C. Groves, and J.
Estes. 2001. Nature reserves: Do they capture the full range of
America's biological diversity?. Ecological Applications
\textbf{11}:999--1007.

Siitonen, J., P. Martikainen, P. Punttila, and J. Rauh. 2000. Coarse
woody debris and stand characteristics in mature managed and old-growth
boreal mesic forests in southern Finland. Forest Ecology and Management
\textbf{128}:211--225. Retrieved from
\url{http://linkinghub.elsevier.com/retrieve/pii/S0378112799001486}.

Sirkiä, S., J. Lehtomäki, H. Lindén, E. Tomppo, and A. Moilanen. 2012.
Defining spatial priorities for capercaillie Tetrao urogallus lekking
landscape conservation in south-central Finland. Wildlife Biology
\textbf{18}:337--353.

Timonen, J., L. Gustafsson, J. S. Kotiaho, and M. Mönkkönen. 2011.
Hotspots in cold climate : Conservation value of woodland key habitats
in boreal forests. Biological Conservation. Retrieved from
\url{http://dx.doi.org/10.1016/j.biocon.2011.02.016}.

Timonen, J., J. Siitonen, L. Gustafsson, J. S. Kotiaho, J. N. Stokland,
A. Sverdrup-Thygeson, and M. Mönkkönen. 2010. Woodland key habitats in
northern Europe: concepts, inventory and protection. Scandinavian
Journal of Forest Research \textbf{25}:309--324. Retrieved from
\url{http://www.tandfonline.com/doi/abs/10.1080/02827581.2010.497160}.

\section{Tables}

\section{Figures}

\textbf{Figure 1.} Schematics of the analysis setup. Different parts of
the analysis were done in different software environments (see text).
Analysis feature layers (index layers) were constructed from 3 different
data sources (Table 1 and a condition layer (see text) was applied on
all of them. Different Zonation analysis variants are indicated by
arrows 1-4 (see Table 3) with closed circles indicating the analysis
features used. Each analysis variant resulted in a priority maps (Figure
2) and feature-specific performance curves (Figure 3). For validation
purposes, each rank priority map was compared to a set of independent
validation data to determine the mean and the distribution of ranks
(Figure 4).

\textbf{Figure 2.} Conservation rank priority maps.

\textbf{Figure 3.} Spatial overlaps of solutions based on different data
sets and variants.

\textbf{Figure 4.} Performance curves for variant XXX (REPLACE
``PERFORMANCE CURVES'')

\textbf{Figure 5.} Distribution of rank priorities in comparison data
sets.

\clearpage

\section{Supplementary material}

Building the ecological model Expert elicitation Figure S1: The benefit
functions used to scale the perceived, expert opinion based conservation
value (y-axis) to forest structural characteristics (x-axis)
Segmentation of the MSNFI data

Analysis setup: Table S1: Feature weights Table S2: Connectivity matrix

\end{document}
