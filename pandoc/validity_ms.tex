\documentclass[]{article}
\usepackage{amssymb,amsmath}
\usepackage{ifxetex,ifluatex}
\usepackage{fixltx2e} % provides \textsubscript
\ifxetex
  \usepackage{fontspec,xltxtra,xunicode}
  \defaultfontfeatures{Mapping=tex-text,Scale=MatchLowercase}
  \newcommand{\euro}{€}
\else
  \ifluatex
    \usepackage{fontspec}
    \defaultfontfeatures{Mapping=tex-text,Scale=MatchLowercase}
    \newcommand{\euro}{€}
  \else
    \usepackage[utf8]{inputenc}
  \fi
\fi
\ifxetex
  \usepackage[setpagesize=false, % page size defined by xetex
              unicode=false, % unicode breaks when used with xetex
              xetex,
              bookmarks=true,
              pdfauthor={},
              pdftitle={},
              colorlinks=true,
              urlcolor=blue,
              linkcolor=blue]{hyperref}
\else
  \usepackage[unicode=true,
              bookmarks=true,
              pdfauthor={},
              pdftitle={},
              colorlinks=true,
              urlcolor=blue,
              linkcolor=blue]{hyperref}
\fi
\hypersetup{breaklinks=true, pdfborder={0 0 0}}
\setlength{\parindent}{0pt}
\setlength{\parskip}{6pt plus 2pt minus 1pt}
\setlength{\emergencystretch}{3em}  % prevent overfull lines
\setcounter{secnumdepth}{0}
\usepackage[vmargin=1in,hmargin=1in]{geometry} 


\begin{document}

\section{Large-scale conservation prioritization in boreal forest
landscapes: generating and validating priority maps}

Joona Lehtomäki\textsuperscript{1,2*} Sakari Tuominen\textsuperscript{3}
and Antti Leinonen\textsuperscript{4}

1 Department of Biosciences, P.O. Box 65 (Viikinkaari 1), FI-00014
University of Helsinki, Finland\\2 Finnish Environment Institute,
Natural Environment Centre, P.O. Box 140 (Mechelininkatu 34a), FI- 00251
Helsinki, Finland\\3 Finnish Forest Research Institute
(Metsäntutkimuslaitos), Vantaa, Finland\\4 The Finnish Forest Centre
(Suomen Metsäkeskus), Kajaani, Finland\\* Corresponding author

\emph{Contact:}\\joona.lehtomaki@helsinki.fi, tel.
+358-9-191-57714\\sakari.tuominen@metla.fi\\antti.leinonen@metsakeskus.fi

\emph{Journal:} Scandinavian Journal of Forest Research

\emph{Type of paper:} Original research paper

\emph{Running title:} Validation of spatial conservation prioritization

\emph{Manuscript statistics:}

\begin{itemize}
\itemsep1pt\parskip0pt\parsep0pt
\item
  word count (total) = xxx
\item
  figures = xxx of which xxx in color
\item
  tables = xxx
\item
  references = xxx
\end{itemize}

\subsection{Abstract:}

Scandinavian boreal forest landscapes are managed primarily for forestry
purposes, but multi-objective planning including conservation of
biodiversity is becoming a more common. Most often no primary
biodiversity data, such as detailed distribution data for many species,
is available for defining conservation priorities over large areas. In
contrast, data are frequently available about forest characteristics
such as standing tree volume, species composition and soil fertility. To
be of value, data has to be 1) relevant for the conservation planning
need (ecological model), 2) spatially extensive enough, 3) at the right
resolution, and 4) available for the planning process. Here, we
demonstrate how to define conservation priorities in Finnish boreal
forest landscapes using Zonation -- a method and software for spatial
conservation prioritization. Conservation priorities are often generated
without explicit considerations on how well the results capture the
distribution of conservation value. Here, we assess the validity of
Zonation results by a comparison to a number of independent data sets
that describe the distributions of features relevant for conservation,
such as existing protected areas and woodland key-habitats. Furthermore,
we investigate what effect the quality of the input forest inventory
data has on the results. We found that prioritization based on data on
forest structure can indeed produce results that are informative for
conservation. Occurrences of independently surveyed features occurred on
average in areas that Zonation had identified as high priority based on
nationally available forest data alone. The process described here and
the results produced by these analyses feed directly into operational
forest management by the Finnish Forestry Center (Metsäkeskus). The
analyses described here could plausibly be implemented also in other
regions of the boreal zone.

Keywords: adaptive management; conservation planning; forest
conservation management; spatial conservation prioritization; Zonation
software Introduction

\subsection{1. Introduction}

\subsubsection{1.1 Operational conservation planning in the forest
context (REF)}

Conservation planning as a part of forestry management: MCDM (Kangas et
al., 2005), TRIAD (Côté et al., 2010), guidelines and policies (Hanski,
2011), new management regimes {[}1{]} Forest conservation planning needs
to integrate closely with the existing planning context and operations
(Ferrier and Wintle, 2009) Successful integration largely depends on
whether existing data that are already part of forestry planning can be
utilized in conservation planning (REF) and on whether the results of
conservation prioritization can be used in tools already existing in
different administrative institutions (REF) In Finland, the
decision-making context is a mixture of top-down and bottom-up action
and different governance processes (Paloniemi and Tikka, 2008)
Introduction to METSO-programme in Finland: aims, schedule, used tools
with emphasis in ESMK (REF) {[}Not sure if needed: then there must be
many different approaches to forest conservation / management: do these
need to be discussed? Perhaps in the discussion - just a note.{]}

\subsubsection{1.2 Spatial conservation prioritization}

Get relevant bits from the workflow MS Has been done in also in the
botreal forest context (Lehtomäki et al., 2009; Mikusiński et al., 2007;
Sirkiä et al., 2012)⁠ Associated uncertainties often high , performance
and efficiency unknown (Langford et al., 2011), need for (on-the-ground)
validation

\subsubsection{1.3 Validation of spatial conservation prioritization
results}

Visually appealing priority maps may influence our perception (as
defined by the ecological model) of the distribution of conservation
value → how to gauge whether the results indeed are useful? The
usefulness of the results also depends on the sensitivity of the
results. Particularly, if the informative part of the results (i.e.~the
best fraction of the landscape for e.g.~conservation) is small relative
to the overall size of the landscape then the results may very sensitive
to various factors

\subsubsection{1.4 Aims and scope of the paper}

\begin{enumerate}
\def\labelenumi{\arabic{enumi}.}
\itemsep1pt\parskip0pt\parsep0pt
\item
  Investigate whether commonly available forestry data sets are a useful
  basis for spatial conservation prioritization The nature of the data
  (MSNFI vs.~more detailed data) Technical usability of the data
  Adequacy of data in the construction of the ecological model
\item
  Investigate how well spatial conservation prioritization (using the
  previous data and Zonation) can inform conservation decision-making in
  operational forestry planning in Finland Comparison of the results to
  a set of independent data sets Matching of planning scales
\item
  Suggest ways to improve use of spatial conservation prioritization
  methods in operational forestry planning Importance of monitoring
  which can also be done as a part of standard forestry operations How
  to improve data? How to improve analyses? Material and Methods
\end{enumerate}

\subsection{2. Material and methods}

\subsubsection{2.1 Data for spatial conservation prioritization and its
validation}

Data used for this purpose must be available across the entire study
area, not from individual locations only. Source of the data matters as
different data sets have different levels of uncertainty etc.
Prioritization data sets (ESMK + LTI inventories + segmented MSNFI)
Brief description of the data used Table 1: Data sets used for the
construction of the ecologically based model and index of conservation
value (see 2.3 for explanation). Validation data Table 2: Additional
data used for the independent validation of results

\subsubsection{2.2 The ecological model}

Short description here, more detailed in the supplement What is: a
conceptual model for conservation prioritization In this context,
``ecological model'' means a conceptual model that converts whatever
data we have into something meaningful from the perspective of
conservation The reasoning for translating the information on forest
structural features to something important for conservation (REF) Data
selection by experts (supl.) Benefit functions (supl.)

Justification behind using the benefit functions Weighting of features
and connectivity (supl.) Based on discussions with the experts + web
questionnaire Figure 1: The ecological model. Data sets, the structure
of the index + the selected connectivity distances and connectivity
interaction types. (similar to (Sirkiä et al., 2012)⁠)

\subsubsection{2.4 Analysis variants}

We selected five analysis variants XXX in order to examine a suite of
scenarios directly relevant for the planning need. Table 3: Analysis
variants. (link to Fig 1)

\subsubsection{2.5 Interpretation of results (prioritity rank maps)}

Spatial prioritization Priority rank maps for the five variants of Table
(3) Representation levels for particular feature groups (by tree spp or
fertility?) for particular top fraction(s) of the landscape → this
particular top fraction could correspond to the overall area objective
for ESMK (will have to check what it is) = frx Comparison of variants
Operationally, how ``robust'' are the results? i.e.~depending on which
variant is used, where are highest priorities? Also, what's the effect
of the ``top fraction of the landscape'' selected (frx + others)?
Overlap and correlations Validation with independent data sets
Distributions of the rank priorities using a given comparison data Other
stats (mean, SD, range?) of priorities within comparison data Comparison
to the data (indexes) that the prioritization is based on (Supl?) 2.6
Comparison between the actual analyses and the MSNFI-only analyses The
selected analysis variants (3, 4, 5, and 7) were run also using just the
MSNFI data Results analyzed like in 2.5

\subsection{3.Results}

\subsubsection{3.1 Spatial priorities}

Figure 2: Priority rank maps for analysis variants (all 4). Figure 3:
representation levels of feature groups (grouped by spp or fertility)
for frx and others Priorities tend to be lower on areas with data source
XXX\ldots{} General patterns are -- naturally -- strongly affected by
the connectivity components included, but this is scale dependent

\subsubsection{3.2 Comparison of variants}

Table 4: Summary statistics on the differences between the variants at
given levels of the top fraction of the landscape: overlap. The results
show that the absolute highest fraction (frx) of the landscape {[}is /
is not{]} relatively dependent to which variant is used. The result is -
of course -- very dependent on the analysis variant used Incorporating
connectivity components will result in tradeoffs: accounting for the
connectivity to existing protected areas aggregate priorities near
conservation areas and draw them away from locations further away

\subsubsection{3.3 Comparison to independent data sets}

Figure 4: Distributions of priority ranks (histograms) in different,
independent data sets. Characteristics of individual data sets
(unsurprisingly) determine which variant should be compared to which
data. E.g. the inventory data collected by the conservation NGOs should
be compared to the variant that accounts for the connectivity to
existing protected areas - because this was a specific objective in data
collection.

\subsubsection{3.4 Comparison to MSNFI-only analysis}

Figure 5: Average priorities in the independent comparison data sets
(barplot).

\subsection{4. Discussion}

Comparison to independent data sets relevant for biodiversity
conservation indicate that known valuable locations indeed emerge from
the analysis with high priorities Data from conventional forestry
planning coupled with a spatial conservation prioritization tool like
Zonation can be used to identify areas of high conservation priorities
Characterize how the validation worked or not with different data sets
Note that results should not be extrapolated beyond the current problem
definition Depending on the prioritization tool used, the analysis and
the results often are static → if and when dynamics are not considered
it is hard to say much about persistence and hence effectiveness.
Formulation of the objectives is thus important and the results should
not extrapolated beyond the scope of the analysis at hand Aligning
planning needs, spatial scale, resolution and the (planning)
decision-making context need careful consideration There is no single
best solution: the most suitable solution will have to be carefully
planned This implies that operational capacity also needs to be in place
in order for the analysis be repeatable and adaptive Learning
institutions, enabling and empowerment important for the long-term
adoption of new techniques (refs Knight etc.) Incorporation of the
results into operational forestry planning -- ``the manager's view''
Antti Opportunities for improvement: Data There are clear differences in
the quality of the data How to improve the data base underlying
prioritizations, noting that data needs to be available across a large
area? What would be ideal data for validation? Methods More realistic
ecological models accounting for X, XX, and XXX Known planning needs
that this analysis does not answer. Enabling input to decision-making
Mainstreaming the methods The road forward

\section{Tables}

\section{Figures}

Figure 1. Schematics of the analysis setup. Different parts of the
analysis were done in different software environments (see text).
Analysis feature layers (index layers) were constructed from 3 different
data sources (Table 1 and a condition layer (see text) was applied on
all of them. Different Zonation analysis variants are indicated by
arrows 1-4 (see Table 3) with closed circles indicating the analysis
features used. Each analysis variant resulted in a priority maps (Figure
2) and feature-specific performance curves (Figure 3). For validation
purposes, each rank priority map was compared to a set of independent
validation data to determine the mean and the distribution of ranks
(Figure 4).

\section{Supplementary material}

Building the ecological model Expert elicitation Figure S1: The benefit
functions used to scale the perceived, expert opinion based conservation
value (y-axis) to forest structural characteristics (x-axis)
Segmentation of the MSNFI data

Analysis setup: Table S1: Feature weights Table S2: Connectivity matrix

1. Kuuluvainen T, Grenfell R (2012) Natural disturbance emulation in
boreal forest ecosystem management --- theories, strategies , and a
comparison with conventional even-aged management. Canadian Journal of
Forest Research 1203: 1185--1203. doi:10.1139/X2012-064

\end{document}
