\section{Figures}

\textbf{Figure 1.} Schematics of the analysis setup. Different parts of
the analysis were done in different software environments (see text).
Analysis feature layers (index layers) were constructed from 3 different
data sources (Table 1 and a condition layer (see text) was applied on
all of them. Different Zonation analysis variants are indicated by
arrows 1-4 (see Table 3) with closed circles indicating the analysis
features used. Each analysis variant resulted in a priority maps (Figure
2) and feature-specific performance curves (Figure 3). For validation
purposes, each rank priority map was compared to a set of independent
validation data to determine the mean and the distribution of ranks
(Figure 4).

\textbf{Figure 2.} Conservation rank priority maps.

\textbf{Figure 3.} Performance curves for variant XXX (REPLACE
``PERFORMANCE CURVES'')

\textbf{Figure 4.} Distribution of rank priorities in comparison data
sets.

\includegraphics{figs/Fig1_w500.png}\\\textbf{Figure 1:} Schematics of
the analysis setup.

\includegraphics{figs/Fig2_w500.png}\\\textbf{Figure 2:} Conservation
rank priority maps.

\includegraphics{figs/Fig3_w600.png}\\\textbf{Figure 3:} Performance
curves.

\includegraphics{figs/Fig4_w600.png}\\\textbf{Figure 4:} Rank histograms
for validation data sets.
